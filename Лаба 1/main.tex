\documentclass[12pt,a4paper]{article}
\usepackage[utf8]{inputenc}
\usepackage[russian]{babel}
\usepackage[left=2.00cm, right=2.00cm, top=2.00cm, bottom=2.00cm]{geometry}
\linespread{1.25}
\usepackage{setspace}
\usepackage{indentfirst}
\setlength{\parindent}{1.25cm}
\let\paragraph\ignorespaces
\usepackage{tabularx}
\usepackage{multirow}
\usepackage{graphicx}
\usepackage{xcolor}

\begin{document}
	
\begin{titlepage}
	
\begin{center}
	\large Университет ИТМО\\[5cm]
	\LARGE Практическая работа №1\\
	\normalsize по дисциплине <<Визуализация и моделирование>>\\[5cm]
\end{center}
\begin{flushright}
		\begin{minipage}{0.6\textwidth}
		\begin{flushleft}
			\large
			\singlespacing 
			\textbf{Автор:} Ефимов Павел Леонидови\\
			\textbf{Поток:} ВИМ 1.1\\
			\textbf{Группа:} К3220\\
			\textbf{Факультет:} ИКТ\\
			\textbf{Преподаватель:} Чернышева А.В.
		\end{flushleft}
	\end{minipage}
\end{flushright}

\vfill

\begin{center}
	{\large Санкт-Петербург, \the\year{ г.}}
\end{center}
 
\end{titlepage}
\normalsize

Датасет: https://www.kaggle.com/tmdb/tmdb-movie-metadata\\
Датасет содержит 5 тысяч фильмов с TMDb. В датасете собраны такие данные как: бюджет, компания производитель, дата выхода, прибыль, средний балл.\\ 
\begin{tabular}{ | l | l | l | }
\hline
Название столбца & Даннные, хранящиеся в столбце & Тип данных \\ \hline
budget & Бюджет фильма & Целое число \\
genres & Жанры & Строка \\
homepage & Сайт фильма & Строка \\
id & Номер фильма & Целое число \\
keywords & Ключевые слова & Строки \\
original\_language & Язык оригинала & Строка \\
original\_title & Оригинальное название & Строка \\
overview & Описание & Строка \\
popularity & Популярность & Число с плавающей точкой \\
production\_companies & Компания производитель & Строка \\
production\_countries & Страна производитель & Строка \\
release\_date & Дата выхода & Дата \\
revenue & Прибыль & Целое число \\
runtime & Время показа на экране & Целое число \\
spoken\_languages & Язык озвучки фильма & Строка \\
status & Статус & Строка \\
tagline & Слоган & Строка \\
title & Название & Строка \\
vote\_average & Средняя оценка & Число с плавающей точкой \\
vote\_count & Количество голосов & Целое число \\
\hline
\end{tabular} \\ \\

\textbf{Задачи решаемые датасетом:}
\begin{enumerate} 
  \item Нахождение самых популярных: серий фильмов, режиссеров, жанров
  \item Определение лучшей даты выхода фильма
  \item Вывод зависимости от средней оценки, жанра и режиссера фильма на прибыль
  \item Нахождение самых популярных, прибыльных студий
\end{enumerate}

\end{document}